% \CheckSum{172}
% \iffalse
%<package>\NeedsTeXFormat{pLaTeX2e}
%<package>\ProvidesPackage{masterthesis}
%<package>  [2013/01/29 v1.0 Style file for thesis]
% \changes{v1.0}{2013/01/29}{Initial release.}
%<*driver>
\documentclass{jltxdoc}
\setcounter{StandardModuleDepth}{1}
\begin{document}
  \DocInput{\jobname.dtx}
\end{document}
%</driver>
% \fi
%
% \GetFileInfo{\jobname.sty}
%
% \title{千葉工大生向け修論スタイルファイル}
% \author{齊藤 潤}
% \maketitle
%
% \section{概要}
% 公式でスタイルファイルを配布してくれないので自作しました.たぶん電情
% 向け.自由に使って構いません.
%
% \TeX とかよくわからないのでツッコミくれると嬉しいです.
%
% \subsection{使い方}
% masterthesis.sty を適用したい \TeX ソースと同じディレクトリに置いて,
% プリアンブルで\textbackslash usepackage\{masterthesis\} とやってくだ
% さい.ドキュメントクラスはjsarticleにしてください.具体的な例は
% Template.texを参照してください.
%
% \StopEventually{}
%
% \section{各マクロの説明}
% \subsection{ページ設定}
%
% \begin{macro}{\Front}
% 表紙,概要,目次等のために,ページ番号をローマ数字に変更します.
%    \begin{macrocode}
\newcommand{\Front}{\pagenumbering{roman}}%
%    \end{macrocode}
% \end{macro}
%
% \begin{macro}{\Main}
% 本文のために,ページ番号をアラビア数字に変更します.
%    \begin{macrocode}
\newcommand{\Main}{\pagenumbering{arabic}}%
%    \end{macrocode}
% \end{macro}
%
% \begin{macro}{\DefaultBaseLineSkip}
% 行送りをデフォルト値に設定します.
% \iffalse
%<*package>
% \fi
%    \begin{macrocode}
\newcommand{\DefaultBaseLineSkip}{\setlength{\baselineskip}{15pt}}%
%    \end{macrocode}
% \end{macro}
%
% \begin{macro}{\DefaultHOffset}
% 左マージンをデフォルト値に設定します.
%    \begin{macrocode}
\newcommand{\DefaultHOffset}{\setlength{\hoffset}{1cm}}%
%    \end{macrocode}
% \end{macro}
%
% \begin{macro}{\DefaultTextWidth}
% 一行あたりの文字数をデフォルト値に設定します.
%    \begin{macrocode}
\newcommand{\DefaultTextWidth}{\setlength{\textwidth}{33zw}}%
%    \end{macrocode}
% \end{macro}
%
% \subsection{論文の情報}
%
% \begin{macro}{\StudentId}
% 学生番号を設定します.
%    \begin{macrocode}
\newcommand{\StudentId}[1]{\global\def\@studentid{#1}}%
%    \end{macrocode}
% \end{macro}
%
% \begin{macro}{\Adviser}
% 指導教員の氏名を設定します.
%    \begin{macrocode}
\newcommand{\Adviser}[1]{\global\def\@adviser{#1}}%
%    \end{macrocode}
% \end{macro}
%
% \begin{macro}{\Major}
% 専攻名を設定します.
%    \begin{macrocode}
\newcommand{\Major}[1]{\global\def\@major{#1}}%
%    \end{macrocode}
% \end{macro}
%
% \begin{macro}{\Title}
% 論文のタイトルを設定します.
%    \begin{macrocode}
\newcommand{\Title}[1]{%
  \global\def\@thetitle{#1}%
  \title{%
    千葉工業大学\\修士学位論文\\%
    \vspace{8mm}%
    \Huge\textbf{\@thetitle\\[8cm]}%
  }%
}%
%    \end{macrocode}
% \end{macro}
%
% \begin{macro}{\Author}
% 論文の著者名を設定します.
%     \begin{macrocode}
\newcommand{\Author}[1]{%
  \global\def\@theauthor{#1}%
  \author{%
    \begin{tabular}{l}%
      所属専攻:\@major\\%
      学生番号・氏名:{\@studentid}番 \@theauthor\\%
      指導教員:{\@adviser}\\\\%
    \end{tabular}%
  }%
}%
%    \end{macrocode}
% \end{macro}
%
% \subsection{ページ作成}
%
% \begin{macro}{\MakeTitlePage}
% 表紙ページを作成します.
%    \begin{macrocode}
\newcommand{\MakeTitlePage}{\maketitle\thispagestyle{empty}\newpage\DefaultBaseLineSkip}%
%    \end{macrocode}
% \end{macro}
%
% \begin{environment}{\Abstract}
% 概要ページを作成します.
%    \begin{macrocode}
\newenvironment{Abstract}{%
  \pagestyle{empty}%
  \begin{center}%
    {\Large\rmfamily\bfseries\@thetitle}%
  \end{center}%
  \normalsize%
  \begin{flushright}%
    \vspace{0.32cm}%
    \@studentid \@theauthor\\%
    \vspace{0.32cm}%
    主査 \@adviser%
    \vspace{0.32cm}%
  \end{flushright}%
  \small\par%
}{%
  \normalsize%
  \newpage%
}%
%    \end{macrocode}
% \end{environment}
%
% \begin{macro}{\TableOfContents}
% 目次ページを作成します.
%    \begin{macrocode}
\newcommand{\TableOfContents}{\setcounter{page}{1}\setcounter{tocdepth}{3}\tableofcontents\newpage}
%    \end{macrocode}
% \end{macro}
%
% \begin{macro}{\Achievements}
% 研究業績のページを作成します.
%    \begin{macrocode}
\newenvironment{Achievements}%
{%
  \let\@temp\refname%
  \renewcommand{\refname}{研究業績}%
  \begin{thebibliography}{99}%
}%
{%
  \end{thebibliography}%
  \renewcommand{\refname}{\@temp}
}
%   \end{macrocode}
% \end{macro}
%
% \begin{macro}{\AchieveItem}
% 研究業績の項目を作成します.\textbackslashbibitemのエイリアスです.
%    \begin{macrocode}
\newcommand{\AchieveItem}[1]{\bibitem{#1}}
%   \end{macrocode}
% \end{macro}
%
% \subsection{既存マクロの変更}
%
% \begin{macro}{\section}
% sectionコマンドをオーバーライドして,セクションごとに改ページするようにします.
%    \begin{macrocode}
\if@twocolumn
\else
\renewcommand{\section}{%
  \newpage
  \if@slide\clearpage\fi
  \@startsection{section}{1}{\z@}%
                {\Cvs \@plus.5\Cdp \@minus.2\Cdp}% 前アキ
                {.5\Cvs \@plus.3\Cdp}% 後アキ
                {\normalfont\Large\headfont\raggedright}}
\fi
%    \end{macrocode}
% \end{macro}
%
% \begin{macro}{\paragraph}
% paragraphコマンドをオーバーライドして,行頭の■なし+行末に改行を入れるようにします.
%    \begin{macrocode}
\if@twocolumn
\else
  \renewcommand{\paragraph}{\@startsection{paragraph}{4}{\z@}%
    {0.5\Cvs \@plus.5\Cdp \@minus.2\Cdp}%
    {\z@}%
    {\normalfont\normalsize\headfont}}
\fi
%    \end{macrocode}
% \end{macro}
%
% \subsection{その他}
% スタイルファイルをロードした後は行送り,左マージン,一行あたりの文字
% 数がデフォルト値にセットされます.
%    \begin{macrocode}
\DefaultBaseLineSkip
\DefaultHOffset
\DefaultTextWidth
%    \end{macrocode}
% \iffalse
%</package>
% \fi
% \Finale
